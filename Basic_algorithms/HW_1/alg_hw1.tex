\documentclass[12pt]{extreport}
\usepackage[T2A]{fontenc}
\usepackage[utf8]{inputenc}        % Кодировка входного документа;
                                    % при необходимости, вместо cp1251
                                    % можно указать cp866 (Alt-кодировка
                                    % DOS) или koi8-r.

\usepackage[english,russian]{babel} % Включение русификации, русских и
                                    % английских стилей и переносов
%%\usepackage{a4}
%%\usepackage{moreverb}
\usepackage{amsmath,amsfonts,amsthm,amssymb,amsbsy,amstext,amscd,amsxtra,multicol}
\usepackage{indentfirst}
\usepackage{verbatim}
\usepackage{tikz} %Рисование автоматов
\usetikzlibrary{automata,positioning}
\usepackage{multicol} %Несколько колонок
\usepackage{graphicx}
\usepackage[colorlinks,urlcolor=blue]{hyperref}
\usepackage[stable]{footmisc}

%% \voffset-5mm
%% \def\baselinestretch{1.44}
\renewcommand{\theequation}{\arabic{equation}}
\def\hm#1{#1\nobreak\discretionary{}{\hbox{$#1$}}{}}
\newtheorem{Lemma}{Лемма}
\theoremstyle{definiton}
\newtheorem{Remark}{Замечание}
%%\newtheorem{Def}{Определение}
\newtheorem{Claim}{Утверждение}
\newtheorem{Cor}{Следствие}
\newtheorem{Theorem}{Теорема}
\theoremstyle{definition}
\newtheorem{Example}{Пример}
\newtheorem*{known}{Теорема}
\def\proofname{Доказательство}
\theoremstyle{definition}
\newtheorem{Def}{Определение}

%% \newenvironment{Example} % имя окружения
%% {\par\noindent{\bf Пример.}} % команды для \begin
%% {\hfill$\scriptstyle\qed$} % команды для \end






%\date{22 июня 2011 г.}
\let\leq\leqslant
\let\geq\geqslant
\def\MT{\mathrm{MT}}
%Обозначения ``ажуром''
\def\BB{\mathbb B}
\def\CC{\mathbb C}
\def\RR{\mathbb R}
\def\SS{\mathbb S}
\def\ZZ{\mathbb Z}
\def\NN{\mathbb N}
\def\FF{\mathbb F}
%греческие буквы
\let\epsilon\varepsilon
\let\es\varnothing
\let\eps\varepsilon
\let\al\alpha
\let\sg\sigma
\let\ga\gamma
\let\ph\varphi
\let\om\omega
\let\ld\lambda
\let\Ld\Lambda
\let\vk\varkappa
\let\Om\Omega
\def\abstractname{}

\def\R{{\cal R}}
\def\A{{\cal A}}
\def\B{{\cal B}}
\def\C{{\cal C}}
\def\D{{\cal D}}

%классы сложности
\def\REG{{\mathsf{REG}}}
\def\CFL{{\mathsf{CFL}}}


%%%%%%%%%%%%%%%%%%%%%%%%%%%%%%% Problems macros  %%%%%%%%%%%%%%%%%%%%%%%%%%%%%%%


%%%%%%%%%%%%%%%%%%%%%%%% Enumerations %%%%%%%%%%%%%%%%%%%%%%%%

\newcommand{\Rnum}[1]{\expandafter{\romannumeral #1\relax}}
\newcommand{\RNum}[1]{\uppercase\expandafter{\romannumeral #1\relax}}

%%%%%%%%%%%%%%%%%%%%% EOF Enumerations %%%%%%%%%%%%%%%%%%%%%

\usepackage{xparse}
\usepackage{ifthen}
\usepackage{bm} %%% bf in math mode
\usepackage{color}
%\usepackage[usenames,dvipsnames]{xcolor}

\definecolor{Gray555}{HTML}{555555}
\definecolor{Gray444}{HTML}{444444}
\definecolor{Gray333}{HTML}{333333}


\newcounter{problem}
\newcounter{uproblem}
\newcounter{subproblem}
\newcounter{prvar}

\def\beforPRskip{
	\bigskip
	%\vspace*{2ex}
}

\def\PRSUBskip{
	\medskip
}


\def\pr{\beforPRskip\noindent\stepcounter{problem}{\bf \theproblem .\;}\setcounter{subproblem}{0}}
\def\pru{\beforPRskip\noindent\stepcounter{problem}{\bf $\mathbf{\theproblem}^\circ$\!\!.\;}\setcounter{subproblem}{0}}
\def\prstar{\beforPRskip\noindent\stepcounter{problem}{\bf $\mathbf{\theproblem}^*$\negthickspace.}\setcounter{subproblem}{0}\;}
\def\prpfrom[#1]{\beforPRskip\noindent\stepcounter{problem}{\bf Задача \theproblem~(№#1 из задания).  }\setcounter{subproblem}{0} }
\def\prp{\beforPRskip\noindent\stepcounter{problem}{\bf Задача \theproblem .  }\setcounter{subproblem}{0} }

\def\prpvar{\beforPRskip\noindent\stepcounter{problem}\setcounter{prvar}{1}{\bf Задача \theproblem \;$\langle${\rm\Rnum{\theprvar}}$\rangle$.}\setcounter{subproblem}{0}\;}
\def\prpv{\beforPRskip\noindent\stepcounter{prvar}{\bf Задача \theproblem \,$\bm\langle$\bracketspace{{\rm\Rnum{\theprvar}}}$\bm\rangle$.  }\setcounter{subproblem}{0} }
\def\prv{\beforPRskip\noindent\stepcounter{prvar}{\bf \theproblem\,$\bm\langle$\bracketspace{{\rm\Rnum{\theprvar}}}$\bm\rangle$}.\setcounter{subproblem}{0} }

\def\prpstar{\beforPRskip\noindent\stepcounter{problem}{\bf Задача $\bf\theproblem^*$\negthickspace.  }\setcounter{subproblem}{0} }
\def\prdag{\beforPRskip\noindent\stepcounter{problem}{\bf Задача $\theproblem^{^\dagger}$\negthickspace\,.  }\setcounter{subproblem}{0} }
\def\upr{\beforPRskip\noindent\stepcounter{uproblem}{\bf Упражнение \theuproblem .  }\setcounter{subproblem}{0} }
%\def\prp{\vspace{5pt}\stepcounter{problem}{\bf Задача \theproblem .  } }
%\def\prs{\vspace{5pt}\stepcounter{problem}{\bf \theproblem .*   }
\def\prsub{\PRSUBskip\noindent\stepcounter{subproblem}{\sf \thesubproblem .} }
\def\prsubr{\PRSUBskip\noindent\stepcounter{subproblem}{\bf \asbuk{subproblem})}\;}
\def\prsubstar{\PRSUBskip\noindent\stepcounter{subproblem}{\rm $\thesubproblem^*$\negthickspace.  } }
\def\prsubrstar{\PRSUBskip\noindent\stepcounter{subproblem}{$\text{\bf \asbuk{subproblem}}^*\mathbf{)}$}\;}

\newcommand{\bracketspace}[1]{\phantom{(}\!\!{#1}\!\!\phantom{)}}

\DeclareDocumentCommand{\Prpvar}{ O{null} O{} }{
	\beforPRskip\noindent\stepcounter{problem}\setcounter{prvar}{1}{\bf Задача \theproblem
% 	\ifthenelse{\equal{#1}{null}}{  }{ {\sf $\bm\langle$\bracketspace{#1}$\bm\rangle$}}
%	~\!\!(\bracketspace{{\rm\Rnum{\theprvar}}}).  }\setcounter{subproblem}{0}
%	\;(\bracketspace{{\rm\Rnum{\theprvar}}})}\setcounter{subproblem}{0}
%
	\,{\sf $\bm\langle$\bracketspace{{\rm\Rnum{\theprvar}}}$\bm\rangle$}
	~\!\!\! \ifthenelse{\equal{#1}{null}}{\!}{{\sf(\bracketspace{#1})}}}.

}
%\DeclareDocumentCommand{\Prpvar}{ O{level} O{meta} m }{\prpvar}


\DeclareDocumentCommand{\Prp}{ O{null} O{null} }{\setcounter{subproblem}{0}
	\beforPRskip\noindent\stepcounter{problem}\setcounter{prvar}{0}{\bf Задача \theproblem
	~\!\!\! \ifthenelse{\equal{#1}{null}}{\!}{{\sf(\bracketspace{#1})}}
	 \ifthenelse{\equal{#2}{null}}{\!\!}{{\sf [\color{Gray444}\,\bracketspace{{\fontfamily{afd}\selectfont#2}}\,]}}}.}

\DeclareDocumentCommand{\Pr}{ O{null} O{null} }{\setcounter{subproblem}{0}
	\beforPRskip\noindent\stepcounter{problem}\setcounter{prvar}{0}{\bf\theproblem
	~\!\!\! \ifthenelse{\equal{#1}{null}}{\!\!}{{\sf(\bracketspace{#1})}}
	 \ifthenelse{\equal{#2}{null}}{\!\!}{{\sf [\color{Gray444}\,\bracketspace{{\fontfamily{afd}\selectfont#2}}\,]}}}.}

%\DeclareDocumentCommand{\Prp}{ O{level} O{meta} }

\DeclareDocumentCommand{\Prps}{ O{null} O{null} }{\setcounter{subproblem}{0}
	\beforPRskip\noindent\stepcounter{problem}\setcounter{prvar}{0}{\bf Задача $\bm\theproblem^* $
	~\!\!\! \ifthenelse{\equal{#1}{null}}{\!}{{\sf(\bracketspace{#1})}}
	 \ifthenelse{\equal{#2}{null}}{\!\!}{{\sf [\color{Gray444}\,\bracketspace{{\fontfamily{afd}\selectfont#2}}\,]}}}.
}

\DeclareDocumentCommand{\Prpd}{ O{null} O{null} }{\setcounter{subproblem}{0}
	\beforPRskip\noindent\stepcounter{problem}\setcounter{prvar}{0}{\bf Задача $\bm\theproblem^\dagger$
	~\!\!\! \ifthenelse{\equal{#1}{null}}{\!}{{\sf(\bracketspace{#1})}}
	 \ifthenelse{\equal{#2}{null}}{\!\!}{{\sf [\color{Gray444}\,\bracketspace{{\fontfamily{afd}\selectfont#2}}\,]}}}.
}


\def\prend{
	\bigskip
%	\bigskip
}




%%%%%%%%%%%%%%%%%%%%%%%%%%%%%%% EOF Problems macros  %%%%%%%%%%%%%%%%%%%%%%%%%%%%%%%



%\usepackage{erewhon}
%\usepackage{heuristica}
%\usepackage{gentium}

\usepackage[portrait, top=3cm, bottom=1.5cm, left=3cm, right=2cm]{geometry}

\usepackage{fancyhdr}
\pagestyle{fancy}
\renewcommand{\headrulewidth}{0pt}
\lhead{\fontfamily{fca}\selectfont {Основные алгоритмы 2020} }
%\lhead{ \bf  {ТРЯП. } Семинар 1 }
%\chead{\fontfamily{fca}\selectfont {Вариант 1}}
\rhead{\fontfamily{fca}\selectfont Домашнее задание 1}
%\rhead{\small 01.09.2016}
\cfoot{}

\usepackage{titlesec}
\titleformat{\section}[block]{\Large\bfseries\filcenter {\setcounter{problem}{0}}  }{}{1em}{}


%%%%%%%%%%%%%%%%%%%%%%%%%%%%%%%%%%%%%%%%%%%%%%%%%%%% Обозначения и операции %%%%%%%%%%%%%%%%%%%%%%%%%%%%%%%%%%%%%%%%%%%%%%%%%%%% 
                                                                    
\newcommand{\divisible}{\mathop{\raisebox{-2pt}{\vdots}}}           
\let\Om\Omega


%%%%%%%%%%%%%%%%%%%%%%%%%%%%%%%%%%%%%%%% Shen Macroses %%%%%%%%%%%%%%%%%%%%%%%%%%%%%%%%%%%%%%%%
\newcommand{\w}[1]{{\hbox{\texttt{#1}}}}

\author{Artem Petrov}
\title{Homework № 1 for "Basic algorithms" course}
\date{{\selectlanguage{english}\today} }
\begin{document}
\begin{minipage}{\textwidth}
	\maketitle
\end{minipage}




                                                                                                       
\pr Верно ли, что \prsubr $ n = O(n\log n)$? \prsubr $\exists \eps > 0 : n\log n = \Om(n^{1+\eps})$?  

$\triangle$

\textbf{a)} $ \exists N = 2 , C = 1: \forall n \in \mathbb{N}, n \geq N: n\leq Cn \log_2 (n) \Rightarrow n = O(n\log_2(n)) \Rightarrow n =O(n \log(n))$

\textbf{b)} From calculus it's known, that $ \forall \epsilon > 0, C>0\; \forall N \in \mathbb{N}, \exists n \in \mathbb{N}, n>N: n^\epsilon > C\log_2(n) $.

Therefore $ \forall \epsilon > 0, C>0\; \forall N \in \mathbb{N}, \exists n \in \mathbb{N}, n>N: n\log_2(n) < Cnn^\epsilon  \Rightarrow \forall \epsilon>0\; n\log(n) \ne \Omega(n^{1+\epsilon})$ 

\hfill$ \square $

\pr Известно, что $f(n) = O(n^2), g(n) = \Omega(1), g(n) = O(n)$. Положим $$h(n) = \cfrac{f(n)}{g(n)}.$$ 

\prsub Возможно ли, что \textbf{а)} $h(n) = \Theta(n\log n)$; \textbf{б)} $h(n) = \Theta(n^3)$ ?

\prsub Приведите наилучшие (из возможных) верхние и нижние оценки на функцию $h(n)$ и приведите пример функций $f(n)$ и $g(n)$ для которых ваши оценки на $h(n)$ достигаются.

$\triangle$

$ g(n) = \Omega(1) \Rightarrow \exists C_g>0, N_g\in \mathbb{N}: \forall n>N g(n)\geq C_g $

$ f(n) = O(n^2) \Rightarrow \exists C_f>0, N_f\in \mathbb{N}: \forall n>N f(n)\leq C_fn^2$

\textbf{1.a)} Yes. $ f(n) = n\log(n), g(n) = 1 \Rightarrow h(n)= n\log(n)$ 

\textbf{1.b)} No.  

$ \forall C_h>0\; \forall N_h \in \mathbb{N}, \exists n = \max(N_h, N_g,N_f, 2\lceil C_f/(C_gC_h)\rceil ) \in \mathbb{N}, n\geq N_h: h(n) = f(n)/g(n) \leq C_f/C_g n^2 \le C_hn^3 \Leftrightarrow \neg (h(n) = \Omega(n^3)$ 

\textbf{2.} Lower bound for h(n) does not exist since there are no lower bound for f(n).

Higher bound for $ h(n) $ is $ O(n^2) $. Here is the proof:

$ \exists C_h = C_f/C_g > 0, \exists N_h = max (N_g,N_f)>0, N \in \mathbb{N}: \forall n > N_h h(n) = f(n)/g(n) \leq C_fn^2/C_g = C_h n^2$

For example, $f(n) = n^2$, $g(n) = 1$.

\hfill $\square$

\pr Найдите $\Theta$-асимптотику $\sum\limits_{i=1}^n \sqrt{i^3+2i+5}$.

$\triangle$

	$ \big] N_0 = 10 $
	$ \forall i > N_0 i^3>2i+5$
	
	$ \exists C_1 = 2 > 0, \exists N_1 = N_0 : \forall n >N_1 \; \sum_{i=1}^{n} \sqrt{i^3 + 2i +5} \leq \sum_{i=1}^{n} \sqrt{n^3 + 2n +5} \leq \sum_{i=1}^{n} \sqrt{2n^3} \leq C_1 n^{5/2} \Rightarrow \sum_{i=1}^{n} \sqrt{i^3 + 2i +5} = O(n^{5/2})$
	
	
	$ \exists C_2 = 2^{5/2} > 0, \exists N_2 = 10 : \forall n >N_2 \; \sum_{i=1}^{n} \sqrt{i^3 + 2i +5} \geq  \sum_{i=1}^{n} \sqrt{i^3} \geq \sum_{i=n/2}^{n} \sqrt{i^3}  \geq \sum_{i=n/2}^{n} \sqrt{(n/2)^3} \geq (n/2)^{5/2} = C_2 n^{5/2} \Rightarrow \sum_{i=1}^{n} \sqrt{i^3 + 2i +5} = \Omega(n^{5/2}) $
	
	$ \sum_{i=1}^{n} \sqrt{i^3 + 2i +5} =  \Omega(n^{5/2}) $ and $ \sum_{i=1}^{n} \sqrt{i^3 + 2i +5} = O(n^{5/2}) \Rightarrow \sum_{i=1}^{n} \sqrt{i^3 + 2i +5} = \Theta(n^{5/2}) $

\hfill $\square$

\pr Пусть для положительной функции $f(n)$ известно, что $f(n) = (3 + o(1))^n + \Theta(n^{100})$. Верно ли в общем случае, что $\log f(n) = \Theta(n)$?
       
       $\triangle$
       
       $ f(n) = (3 + h(n)^n + g(n)$
       
       $ \lim\limits_{n\rightarrow\infty} \frac{h(n)}{1} = 0 \Rightarrow \exists N_h\in \mathbb{N} : \forall n> N_h, n \in \mathbb{N} \; |h(n)| < 1$
       
       $ \exists C>0,c>0, \exists N_g\in \mathbb{N} : \forall n > N_g \; cn^{100}\leq g(n) \leq Cn^{100}$
       
       From calculus we know, that $ \exists N_C \in \mathbb{N} : \forall n \in \mathbb{N} , n > N_C \; 4^n>Cn^{100}$
       
       $ \exists C_1=\log(2) >0,c_1=2\log(4) >0, \exists N_f = max(N_g, N_h, N_C, 10 )\in \mathbb{N} : \forall n > N_f \;c_1n=2\log(4)n\geq\log(2) + n\log(4)=\log(2*4^n) \geq \log((4)^n + Cn^{100})\geq \log((3+h(n))^n + g(n) \geq \log(2^n + cn^{100}) \geq \log(2^n) = C_1n \Rightarrow f(n)=\Theta(n)$
       
       
       
       \hfill $\square$
       
\pr Дана программа

{\tt for} (bound = 1; bound * bound < n; bound += 1 ) \{

 \hspace{4mm} {\tt for} (i = 0; i < bound; i += 1) \{

 \hspace{8mm} {\tt for} (j = 0; j < i; j += 2)
 
 \hspace{12mm} {\tt печать} (``алгоритм'')

 \hspace{8mm} {\tt for} (j = 1; j < n; j *= 2)

 \hspace{12mm} {\tt печать} (``алгоритм'')
 
 \hspace{4mm} \}

\}

\smallskip

\noindent Пусть $g(n)$ обозначает число слов ``алгоритм'', которые напечатает программа. Найдите $\Theta$--асимптотику $g(n)$.

$\triangle$
	
	$g(n) = \sum\limits_{b=1}^{\sqrt{n}} \sum\limits_{i=0}^{b-1}(\sum\limits_{j=0}^{i/2 - 1}1 + \sum\limits_{j=1}^{\log_2n}1) = 
	\sum\limits_{b=1}^{\sqrt{n}} \sum\limits_{i=0}^{b-1}(i/2+\log_2n) = \sum\limits_{b=1}^{\sqrt{n}}((b-1)b/4+b\log_2n) = \Theta(n^{3/2}) + \Theta(n\log n) \Rightarrow g(n) = \Theta(n^{3/2})$
	%(\sqrt{n})(\sqrt{n}-1)/2 * \log_2n + 1/4(n^(3/2)/3 -  = \Theta(n\log(n))$
	
\hfill $\square$

\end{document}
  
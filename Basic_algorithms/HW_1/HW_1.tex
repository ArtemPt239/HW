% !TeX program = lualatex

\documentclass[a4paper,12pt]{article}


\usepackage[T1]{fontenc}			% кодировка
\usepackage[utf8]{inputenc}			% кодировка исходного текста
%\usepackage[lutf8x]{luainputenc}
%\usepackage[english,russian]{babel}	% локализация и переносы
\usepackage[english,russian]{babel}
\usepackage{odsfile}
% Математика
\usepackage{amsmath,amsfonts,amssymb,amsthm,mathtools}

\usepackage{gensymb}
\usepackage{wasysym}

% Шрифты и текст
\usepackage{bm}

% Картинки
\usepackage{graphicx}
\usepackage{subcaption}
\graphicspath{{images/}}

%Заговолок
\usepackage[left=2cm,right=2cm,
top=2cm,bottom=2cm,bindingoffset=0cm]{geometry}

\usepackage{titling}


\newcommand{\Mod}[1]{\ (\mathrm{mod}\ #1)}
\newcommand{\gmznz}[1]{(\mathbb{Z}/#1\mathbb{Z})^*}

\author{Artem Petrov}
\title{Homework № 1 for "Basics of applied algebra and coding theory" course}
\date{{\selectlanguage{english}\today} }

\begin{document} % начало документа

\maketitle

\section*{Problem \#2}

\vartriangle

Let's select arbitrary $ z \in G$.
\[ z = z z z^{-1} = e z^{-1} = z^{-1}\]

So, $ \forall z \in G: z = z^{-1} $.
Let's select arbitrary $ x, y \in G$.
\[ xy = x^{-1}y^{-1} = (yx)^{-1} = yx \]

\hfill \square

\section*{Problem \#3}

\triangle

According to the definition of group $ \forall g \in G,  \exists! g^{-1} \in G : gg^{-1} = e $.
\[ g^{-1} = g \Leftrightarrow  g^{2} = e, \]

so if $ g\ne f $ (and $ g \ne e $), than $ g \ne g^{-1} $. (otherwise we would have $ g^{2} = gg^{-1} = e \Rightarrow |g| = 2$, which is forbidden by problem condition).

\[ \prod_{g \in G} g = e f g_{1}g_{1}^{-1}g_{2}g_{2}^{-1}... = f\]
\hfill \square

\section*{Problem \#4}

\triangle

Suppose $ |h| = m $.

To proof that $ |ghg^{-1}| = m $ we should prove that $ m = min \{x \in \mathbb{N}: (ghg^{-1})^{x} = e \}$. Let's do this:

1.1) $ (ghg^{-1})^{m} = gh^{m}g^{-1} = geg^{-1} = e$.

1.2) Suppose we have found $ k\in \mathbb{N}: k < m , (ghg^{-1})^{k} = e$. Then, $ e = (ghg^{-1})^{k} = gh^{k}g^{-1} \Rightarrow g^{-1} = h^{k}g^{-1} \Rightarrow e = h^{k} $, which is impossible since $ |h| = m > k $.

So, we have proved that $ m = |ghg^{-1}|$

Moving on to the next question:

Suppose $ |gh| = m $. We will prove that $ |hg| = m $ following the same scheme we have followed in the previous proof:

2.1) $ e = (hg)^{m} = h(gh)^{m-1}g \Leftrightarrow e = geg^{-1} = gh(gh)^{m-1}gg^{-1}= (gh)^m$

2.2)  Suppose we have found $ k\in \mathbb{N}: k < m , (hg)^{k} = e$. Then, similarly to 2.2, we get $ (gh)^k = e $, which is impossible since $ |h| = m > k $.

So, we have proved that $ |gh| = m = |hg|$
\hfill \square

\section*{Problem \#6}

\triangle

$\big] X = \{ z \in \mathbb{C} : z^n = 1 \} = \{ exp(i2\pi k / n): k \in \{0,1,2...(n-1)\} \}$

Clearly, $ (X; *) \cong (\{0,1,2...(n-1)\}, +) $. And therefore $ (X, *) $ is a group.

%		Let's prove that $ (X,*) $ is a group. Let $ x,y,z \in X $.
%		
%		1) $ (xy)z $

$ \forall x \in X \exists y \in X : y^3 = x \Leftrightarrow \forall k \in \{0,1...n-1\} \exists m \in \{0,1...n-1\} : m*3 = k $.

$ \big] n = 35, k\in \{0,1...n-1\}$.

In this case if $ k\equiv 0 \Mod{3} $, then $ m = k/3 $. If $ k\equiv 1 \Mod{3} $, then $ m = (k+35)/3 $. If $ k\equiv2 \Mod{3} $, then $ m = (k+2*35)/3 $. We found corresponding m for every $ k\in X $. Hence, every $ x \in X (n=35) $ is a cube.

$ \big] n = 36, k\in \{0,1...n-1\}: k\equiv 1 \Mod{3}$. Let's prove by contradiction. Assume, $ \exists m\in \{0,1...n-1\}: m*3 = k$. Then $ m*3 \equiv k + 36*l \Mod{3}; (l\in\mathbb{N}) \Rightarrow 0 \equiv 1 \Mod{3}$, which is impossible. Therefore not every $ x \in X (n=36) $ is a cube.

\hfill \square

\section*{Problem \#7}

\triangle

$ \gmznz{5} = (\{1,2,3,4\}, *); \gmznz{12} = (\{1,5,7,11\})$. Clearly, they have the same amount of elements.

Let's build multiplication tables for both of these groups:

\begin{table}[h!]
	\begin{minipage}{0.48\textwidth}
		\centering
		\caption{Multiplication table for $ \gmznz{5} $}
		\begin{tabular}{||c||c|c|c|c||}
			\hline
			\# & 1 & 2 & 3 & 4 \\ 
			\hline\hline
			1  & 1 & 2 & 3 & 4 \\
			2  & 2 & 4 & 1 & 3 \\
			3  & 3 & 1 & 4 & 2 \\
			4  & 4 & 3 & 2 & 1 \\
			\hline
		\end{tabular}

		\label{table:1}
	\end{minipage}
	\begin{minipage}{0.48\textwidth}
		\centering
		\caption{Multiplication table for $ \gmznz{12} $}
		\begin{tabular}{||c||c|c|c|c||}
			\hline
			\# & 1 & 5 & 7 & 11 \\
			\hline\hline
			1  & 1 & 5 & 7 & 11 \\
			5  & 5 & 1 & 11 & 7 \\
			7  & 7 & 11 & 1 & 5 \\
			11  & 11 & 7 & 5 & 1 \\
			\hline
		\end{tabular}

		\label{table:1}
	\end{minipage}
\end{table}

Let's prove by contradiction. Suppose $ \exists \phi : \gmznz{5} \rightarrow \gmznz{12} $ - isomorphism. Isomorphism between groups maps 1 to 1. $ \forall x \in \gmznz{12}\; x^2 = 1  $. Then $\forall y \in \gmznz{5}\; \phi(y*y) = \phi(y)\phi(y) = 1  \Rightarrow y*y = 1$, which is impossible, since $ 2^2 = 4 $ in $\gmznz{5}$. Therefore there are no isomorphisms between this groups.

\hfill \square


\end{document} % конец документа
